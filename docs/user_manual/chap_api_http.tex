\section{Hypertext Transfer Protocol (HTTP)}
TODO

\subsection{pico\_http\_client\_open}

\subsubsection*{Description}
Opens a new connection to a HTTP-server.

\subsubsection*{Function prototype}
\texttt{int32\_t pico\_http\_client\_open(char\_t *hostname,
                void (*wakeup)(uint16\_t ev, uint16\_t conn));}

\subsubsection*{Parameters}
\begin{itemize}[noitemsep]
\item \texttt{hostname} - String specifying the name and port of the HTTP-server to connect to. [http:]hostname[:port]/
\item \texttt{wakeup} - Pointer to a callback function of type \texttt{int fn(uint16\_t, uint16\_t)}.
\end{itemize}

\subsubsection*{Possible events passed to the \texttt{wakeup} function}
\begin{itemize}[noitemsep]
\item \texttt{EV\_HTTP\_CON} - HTTP-client is connected to the HTTP-server.
\item \texttt{EV\_HTTP\_REQ} - Header has arrived.
\item \texttt{EV\_HTTP\_BODY} - Body data available for reading.
\item \texttt{EV\_HTTP\_WRITE\_SUCCESS} - Request was successfully written to the socket.
\item \texttt{EV\_HTTP\_WRITE\_PROGRESS\_MADE} - If it wasn't possible to write everything at once to the socket, this event will come trough to tell write progress has been made.
\item \texttt{EV\_HTTP\_WRITE\_FAILED} - The request was not successfully written to the socket.
\item \texttt{EV\_HTTP\_ERROR} - An error occured.
\item \texttt{EV\_HTTP\_CLOSE} - Close the connection.
\item \texttt{EV\_HTTP\_DNS} - DNS query was successful.
\end{itemize}
\subsubsection*{Return value}
On success connection id \texttt{conn}.
\\On failure \texttt{HTTP\_RETURN\_ERROR}.

\subsubsection*{Example}
\begin{verbatim}
ret = pico_http_client_open(hostname, wakeup_function);
\end{verbatim}

%-----------------------------------------------------------------------------------------------------

\subsection{pico\_http\_client\_close}

\subsubsection*{Description}
Closes the connection and everything related to this connection id will be cleaned up.

\subsubsection*{Function prototype}
\texttt{int8\_t pico\_http\_client\_close(uint16\_t conn);}

\subsubsection*{Parameters}
\begin{itemize}[noitemsep]
\item \texttt{conn} - Connection id.
\end{itemize}
\subsubsection*{Return value}
On success \texttt{HTTP\_RETURN\_OK}.
\\On failure \texttt{HTTP\_RETURN\_ERROR}.
\subsubsection*{Example}
\begin{verbatim}
ret = pico_http_client_close(connection_id);
\end{verbatim}

%-----------------------------------------------------------------------------------------------------

\subsection{pico\_http\_client\_send\_raw}

\subsubsection*{Description}
Makes it possible to send a self made request, nothing will be added or removed from your \texttt{request}.
Make sure \texttt{request} is available untill the event \texttt{EV\_HTTP\_WRITE\_SUCCESS} is passed to the wakeup\_function. When the complete request has been send, \texttt{EV\_HTTP\_WRITE\_SUCCESS} is passed.

\subsubsection*{Function prototype}
\texttt{int8\_t pico\_http\_client\_send\_raw(uint16\_t conn, char *request);}

\subsubsection*{Parameters}
\begin{itemize}[noitemsep]
\item \texttt{conn} - Connection id through which the \texttt{request} has to be send.
\item \texttt{request} - The raw, self build, request that has to be send.
\end{itemize}

\subsubsection*{Return value}
On success \texttt{HTTP\_RETURN\_OK}.
\\On failure \texttt{HTTP\_RETURN\_ERROR}.

\subsubsection*{Example}
\begin{verbatim}
ret = pico_http_client_send_raw(connection_id, request);
\end{verbatim}

%-----------------------------------------------------------------------------------------------------

\subsection{pico\_http\_client\_send\_get}
\subsubsection*{Description}
Send a GET request to the HTTP-server. The library will build the GET request based on the \texttt{resource} and the \texttt{hostname} that was passed on opening the connection to the HTTP-server. Via \texttt{connection\_type} you can select a "Close" or "Keep-Alive" connection. When the complete request has been send, \texttt{EV\_HTTP\_WRITE\_SUCCESS} is passed to the wakeup\_function.

\subsubsection*{Function prototype}
\texttt{int8\_t pico\_http\_client\_send\_get(uint16\_t conn, char *resource, uint8\_t connection\_type);}

\subsubsection*{Parameters}
\begin{itemize}[noitemsep]
\item \texttt{conn} - Connection id through which the GET request has to be send.
\item \texttt{resource} - Path to resource that needs to be addressed.
\item \texttt{connection\_type} - When \texttt{HTTP\_CONN\_KEEP\_ALIVE} is passed, the connection will not be closed and a new request can be send over the same connection. The connection will be closed after the response when \texttt{HTTP\_CONN\_CLOSE} is passed.
\end{itemize}

\subsubsection*{Return value}
On success \texttt{HTTP\_RETURN\_OK}.
\\On failure \texttt{HTTP\_RETURN\_ERROR}.

\subsubsection*{Example}
\begin{verbatim}
ret = pico_http_client_send_get(connection_id, "/test/index.html", HTTP_CONN_CLOSE);
\end{verbatim}

%-----------------------------------------------------------------------------------------------------

\subsection{pico\_http\_client\_send\_post}

\subsubsection*{Description}
Send a POST request to the HTTP-server. The library will build the POST request based on the parameters passed and the \texttt{hostname} that was passed on opening the connection to the HTTP-server. Make sure \texttt{post\_data} is available untill the event \texttt{EV\_HTTP\_WRITE\_SUCCESS} is passed to the wakeup\_function.

\subsubsection*{Function prototype}
\texttt{int8\_t pico\_http\_client\_send\_post(uint16\_t conn, char *resource, uint8\_t *post\_data, uint32\_t post\_data\_len, uint8\_t connection\_type, char *content\_type, char *cache\_control);}

\subsubsection*{Parameters}
\begin{itemize}[noitemsep]
\item \texttt{conn} - Connection id through which the POST request has to be send.
\item \texttt{resource} - Path to resource that needs to be addressed.
\item \texttt{post\_data} - A char buffer with the data to send. Format: "key1=value1\&key2=value2\&..."
\item \texttt{post\_data\_len} - Length of the \texttt{post\_data} in number of bytes.
\item \texttt{connection\_type} - When \texttt{HTTP\_CONN\_KEEP\_ALIVE} is passed, the connection will not be closed and a new request can be send over the same connection. The connection will be closed after the response when \texttt{HTTP\_CONN\_CLOSE} is passed.
\item \texttt{content\_type} - String to specify the content type. If NULL, "application/x-www-form-urlencoded" is added to the header.
\item \texttt{cache\_controle} - String to specify the cache controle. If NULL, "private, max-age=0, no-cache" is added to the header.
\end{itemize}
\subsubsection*{Return value}
On success \texttt{HTTP\_RETURN\_OK}.
\\On failure \texttt{HTTP\_RETURN\_ERROR}.
\subsubsection*{Example}
\begin{verbatim}
char post_data* = "name=dmx&number=10"
ret = pico_http_client_send_post(conn, "/post_test/", post_data, strlen(post_data),
HTTP_CONN_CLOSE, NULL, NULL);
\end{verbatim}

%-----------------------------------------------------------------------------------------------------

\subsection{pico\_http\_client\_send\_delete}

\subsubsection*{Description}
Send a DELETE request to the HTTP-server. The library will build the DELETE request based on the \texttt{resource} and the \texttt{hostname} that was passed on opening the connection to the HTTP-server. Via \texttt{connection\_type} you can select a "Close" or "Keep-Alive" connection. When the complete request has been send, \texttt{EV\_HTTP\_WRITE\_SUCCESS} is passed to the wakeup\_function.

\subsubsection*{Function prototype}
\texttt{int8\_t pico\_http\_client\_send\_delete(uint16\_t conn, char *resource, uint8\_t connection\_type);}

\subsubsection*{Parameters}
\begin{itemize}[noitemsep]
\item \texttt{conn} - Connection id through which the DELETE request has to be send.
\item \texttt{resource} - Path to resource that needs to be addressed.
\item \texttt{connection\_type} - When \texttt{HTTP\_CONN\_KEEP\_ALIVE} is passed, the connection will not be closed and a new request can be send over the same connection. The connection will be closed after the response when \texttt{HTTP\_CONN\_CLOSE} is passed.
\end{itemize}
\subsubsection*{Return value}
On success \texttt{HTTP\_RETURN\_OK}.
\\On failure \texttt{HTTP\_RETURN\_ERROR}.
\subsubsection*{Example}
\begin{verbatim}
ret = pico_http_client_send_delete(connection_id, "/test_delete", HTTP_CONN_CLOSE);
\end{verbatim}

%-----------------------------------------------------------------------------------------------------

\subsection{multipart\_chunk\_create}

\subsubsection*{Description}
Function to create multipart chunks to send via MULTIPART POST request. 

\subsubsection*{Function prototype}
\texttt{struct multipart\_chunk *multipart\_chunk\_create(uint8\_t *data, uint64\_t length\_data, char *name, char *filename, char *content\_disposition, char *content\_type);}

\subsubsection*{Parameters}
\begin{itemize}[noitemsep]
\item \texttt{data} - Pointer to the data that needs to be send.
\item \texttt{length\_data} - Number of bytes in \texttt{data}.
\item \texttt{name} - Name of the data.
\item \texttt{filename} - Name of the file.
\item \texttt{content\_disposition} - For example "attachment". If \texttt{content\_disposition} is used, it would result in: "Content-Disposition: attachment; name="name"; filename="filename"
\item \texttt{content\_type} - The type of content, for example "text/plain"
\end{itemize}
\subsubsection*{Return value}
On failure \texttt{NULL}.
\\On success a pointer to a new \texttt{struct multipart\_chunk}.
\subsubsection*{Example}
\begin{verbatim}
struct multipart_chunk **chunks;
uint32_t chunks_len = 3;
int32_t fd = 0;
off_t buff_size;
uint8_t *buff = NULL;
int32_t rv = 0;

chunks = PICO_ZALLOC(chunks_len * sizeof(struct multipart_chunk *));
chunks[0] = multipart_chunk_create("data1", 5u, "name1", "filename1",
                                   "attachment", NULL);
chunks[1] = multipart_chunk_create("data2", 5u, "name2", "filename2",
                                   "attachment", NULL);
fd=open("./example.jpg", O_RDWR);
buff_size = lseek(fd, 0, SEEK_END);
lseek(fd, 0, SEEK_SET);
buff = PICO_ZALLOC(buff_size);
rv = read(fd, buff, buff_size);
chunks[3] = multipart_chunk_create(buff, buff_size, "name3",
                                   "example.jpg", "attachment", NULL);
\end{verbatim}


%-----------------------------------------------------------------------------------------------------

\subsection{pico\_http\_client\_send\_post\_multipart}

\subsubsection*{Description}
Send a MULTIPART POST to the HTTP-server. The library will build the MULTIPART POST request based on the \texttt{resource} and the \texttt{hostname} that was passed on opening the connection. Via \texttt{connection\_type} you can select a "Close" or "Keep-Alive" connection. Pass the files/data that need to be send via \texttt{post\_data}. Make sure that the data in the multipart chunks is available utill the complete request has been send, \texttt{EV\_HTTP\_WRITE\_SUCCESS} is passed to the wakeup\_function.

\subsubsection*{Function prototype}
\texttt{int8\_t pico\_http\_client\_send\_post\_multipart(uint16\_t conn, char *resource, struct multipart\_chunk **post\_data, uint16\_t post\_data\_len, uint8\_t connection\_type);}

\subsubsection*{Parameters}
\begin{itemize}[noitemsep]
\item \texttt{conn} - Connection id through which the MULTIPART POST request has to be send.
\item \texttt{resource} - Path to the resource that needs to be addressed.
\item \texttt{connection\_type} - When \texttt{HTTP\_CONN\_KEEP\_ALIVE} is passed, the connection will not be closed and a new request can be send over the same connection. The connection will be closed after the response when \texttt{HTTP\_CONN\_CLOSE} is passed.
\item \texttt{post\_data} - A pointer to \texttt{struct multipart\_chunk} pointers.
\item \texttt{post\_data\_len} - Number of \texttt{struct multipart\_chunk} pointers in \texttt{post\_data}.
\end{itemize}
\subsubsection*{Return value}
On success \texttt{HTTP\_RETURN\_OK}.
\\On failure \texttt{HTTP\_RETURN\_ERROR}.
\subsubsection*{Example}
\begin{verbatim}
ret = pico_http_client_send_post_multipart(conn, "/", chunks,
                                           chunks_len, HTTP_CONN_KEEP_ALIVE);
\end{verbatim}

%-----------------------------------------------------------------------------------------------------

\subsection{multipart\_chunk\_destroy}

\subsubsection*{Description}
Function to destroy/free a multipart chunk. The pointer to the data in the chunk is also freed.

\subsubsection*{Function prototype}
\texttt{int8\_t multipart\_chunk\_destroy(struct multipart\_chunk *chunk);}

\subsubsection*{Parameters}
\begin{itemize}[noitemsep]
\item \texttt{chunk} - Pointer to a \texttt{struct multipart\_chunk}.
\end{itemize}
\subsubsection*{Return value}
On failure \texttt{HTTP\_RETURN\_ERROR}.
\\On success \texttt{HTTP\_RETURN\_OK};
\subsubsection*{Example}
\begin{verbatim}
ret = multipart_chunk_destroy(chunk);
\end{verbatim}


%-----------------------------------------------------------------------------------------------------

\subsection{pico\_http\_client\_long\_poll\_send\_get}

\subsubsection*{Description}
Will start LONG POLLING GET request. The library will build the GET request based on the \texttt{resource} and the \texttt{hostname} that was passed on opening the connection. 
\begin{enumerate}
\item Client sends a request.
\item Client waits for the server to respond.
\item Server responds and the wakeup\_function is triggered.
\item When client has read the responsei, the libraby automaticly sends a new GET request.
\end{enumerate}
Dispite which \texttt{connection\_type} you pass, the connection id will be reused for the next GET request.
\subsubsection*{Function prototype}
\texttt{int8\_t pico\_http\_client\_send\_get(uint16\_t conn, char *resource, uint8\_t connection\_type);}

\subsubsection*{Parameters}

\begin{itemize}[noitemsep]
\item \texttt{conn} - Connection id on which the LONG POLLING GET request has to be started.
\item \texttt{resource} - Path to the resource that needs to be addressed.
\item \texttt{connection\_type} - When \texttt{HTTP\_CONN\_KEEP\_ALIVE} is passed, the connection will not be closed and a new request will be send over the same connection. The connection will be close after the response when \texttt{HTTP\_CONN\_CLOSE} is passed and a new one will be created to send the next GET request.
\end{itemize}
\subsubsection*{Return value}
On failure \texttt{HTTP\_RETURN\_ERROR}.
\\On success \texttt{HTTP\_RETURN\_OK};
\subsubsection*{Example}
\begin{verbatim}
ret = pico_http_client_long_poll_send_get(conn, "/", HTTP_CONN_CLOSE);
\end{verbatim}

%-----------------------------------------------------------------------------------------------------

\subsection{pico\_http\_client\_long\_poll\_cancel}

\subsubsection*{Description}
Cancel a running LONG POLLING GET.

\subsubsection*{Function prototype}
\texttt{int8\_t pico\_http\_client\_long\_poll\_cancel(uint16\_t conn);}

\subsubsection*{Parameters}
\begin{itemize}[noitemsep]
\item \texttt{conn} - Connection id
\end{itemize}
\subsubsection*{Return value}
On failure \texttt{HTTP\_RETURN\_ERROR}.
\\On success \texttt{HTTP\_RETURN\_OK};
\subsubsection*{Example}
\begin{verbatim}
ret = pico_http_client_long_poll_cancel(conn);
\end{verbatim}

%-----------------------------------------------------------------------------------------------------

\subsection{pico\_http\_client\_get\_write\_progress}

\subsubsection*{Description}
Function to get the write progress if the library is sending a request.

\subsubsection*{Function prototype}
\texttt{int8\_t pico\_http\_client\_get\_write\_progress(uint16\_t conn, uint32\_t *total\_bytes\_written, uint32\_t *total\_bytes\_to\_write);}

\subsubsection*{Parameters}
\begin{itemize}[noitemsep]
\item \texttt{conn} - Connection id
\item \texttt{total\_bytes\_written} - Pointer to \texttt{uint32\_t} to store the bytes that allready have been send.
\item \texttt{total\_bytes\_to\_write} - (optional) pointer to  \texttt{unint32\_t} to store the total number of bytes that need to be send.
\end{itemize}
\subsubsection*{Return value}
On failure \texttt{HTTP\_RETURN\_ERROR}.
\\On success \texttt{HTTP\_RETURN\_OK};
\subsubsection*{Example}
\begin{verbatim}
int32_t total_bytes_written = 0;
int32_t total_bytes_to_write = 0;
ret = pico_http_client_get_write_progress(conn, &total_bytes_written,
                                          &total_bytes_to_write);
\end{verbatim}

%-----------------------------------------------------------------------------------------------------

\subsection{pico\_http\_client\_read\_header}

\subsubsection*{Description}
Function to get the header info after \texttt{EV\_HTTP\_REQ} was triggered via the callback.

\subsubsection*{Function prototype}
\texttt{struct pico\_http\_header *pico\_http\_client\_read\_header(uint16\_t conn);}

\subsubsection*{Parameters}
\begin{itemize}[noitemsep]
\item \texttt{conn} - Connection id
\end{itemize}

\subsubsection*{Return value}
On failure \texttt{NULL}.
\\On success a pointer to a \texttt{struct pico\_http\_header}.

\subsubsection*{Example}
\begin{verbatim}
struct pico_http_header *header = NULL;
header = pico_http_client_read_header(conn);
printf("Received header from server...\n");
printf("Server response : %d\n",header->response_code);
printf("Location : %s\n",header->location);
printf("Transfer-Encoding : %d\n",header->transfer_coding);
printf("Size/Chunk : %d\n",header->content_length_or_chunk);
\end{verbatim}

%-----------------------------------------------------------------------------------------------------

\subsection{pico\_http\_client\_read\_uri\_data}

\subsubsection*{Description}
Function to get the uri info.

\subsubsection*{Function prototype}
\texttt{struct pico\_http\_uri *pico\_http\_client\_read\_uri\_data(uint16\_t conn);}

\subsubsection*{Parameters}
\begin{itemize}[noitemsep]
\item \texttt{conn} - connection id
\end{itemize}

\subsubsection*{Return value}
On failure \texttt{NULL}.
\\On success a pointer to a \texttt{struct pico\_http\_uri}.

\subsubsection*{Example}
\begin{verbatim}
struct pico_http_uri *uri = NULL;
uri = pico_http_client_read_uri_data(conn);
\end{verbatim}

%-----------------------------------------------------------------------------------------------------

\subsection{pico\_http\_client\_read\_body}

\subsubsection*{Description}
Function to read the body of the response, should be called after \texttt{EV\_HTTP\_BODY} event.

\subsubsection*{Function prototype}
\texttt{int32\_t pico\_http\_client\_read\_body(uint16\_t conn, uint8\_t *data, uint16\_t size, uint8\_t *body\_read\_done);}

\subsubsection*{Parameters}
\begin{itemize}[noitemsep]
\item \texttt{conn} - Connection id
\item \texttt{data} - Pointer to a buffer where the read body data needs to be stored.
\item \texttt{size} - Number of bytes to read.
\item \texttt{body\_read\_done} - Pointer to \texttt{uint8\_t}. Will be set to 1 if the body read is complete.
\end{itemize}
\subsubsection*{Return value}
On failure \texttt{HTTP\_RETURN\_ERROR}.
\\On success the number of bytes that have been read.
\subsubsection*{Example}
\begin{verbatim}
bytes_read = pico_http_client_read_body(conn, &data[_length],1024, &body_read_done);
\end{verbatim}
